%
% Documento: Introdução
%
\chapter{INTRODUÇÃO}\label{chap:introducao}

Apresentar brevemente o documento.

\section{Problemática}

Apresentar dados que descrevam o problema que sua solução pretende resolver, em que linha de pesquisa atua. Dar preferência por dados oficiais, outros trabalhos relacionados. 

\section{Descrição da solução}

Apresentar, de maneira sucinta, a solução proposta. Descrever o que faz, comparar com outras ferramentas semelhantes.

\section{Objetivos}

\subsection{Objetivo Geral}

Descrever em um parágrafo qual o objetivo do seu trabalho.


\subsection{Objetivos Específicos}

Descrever quais objetivos secundários devem ser atingidos para chegar no objetivo geral.

\begin{itemize}
    \item Objetivo 1
    \item Objetivo 2
    \item Objetivo 3
\end{itemize}

\section{Atividades}

Descrever quais serão as atividades a serem desenvolvidas no trabalho. Lembrar que não há somente a parte de desenvolvimento, as tarefas de pesquisar, escrever, entram aqui.

\section{Cronograma}

Apresentar o cronograma das atividades, especialmente para o TCC 2, pois é uma forma de mostrar a viabilidade. Pode organizar as atividades em meses.

\section{Organização do documento}

Apresentar quais são os capítulos seguintes do trabalho e o que é abordado em cada um deles.